\documentclass{amsart}

\usepackage[colorlinks=true]{hyperref}
\usepackage{enumerate}
\usepackage{color}
\usepackage{mathrsfs}
\usepackage{tikz-cd}
\usepackage{amssymb}
\usepackage{centernot}

\usepackage{marginnote}
\renewcommand*{\marginfont}{\scriptsize\color{red}\sffamily}

\theoremstyle{plain}
\newtheorem{theorem}{Theorem}
\newtheorem{lemma}[theorem]{Lemma}
\newtheorem{proposition}[theorem]{Proposition}
\newtheorem{corollary}[theorem]{Corollary}

\theoremstyle{definition}
\newtheorem{definition}[theorem]{Definition}
\newtheorem{example}[theorem]{Example}
\newtheorem{exercise}[theorem]{Exercise}

\theoremstyle{remark}
\newtheorem{remark}[theorem]{Remark}

% Fonts
\newcommand{\A}{\mathbb{A}}
\newcommand{\C}{\mathbb{C}}
\newcommand{\F}{\mathbb{F}}
\newcommand{\R}{\mathbb{R}}
\newcommand{\Q}{\mathbb{Q}}
\newcommand{\Z}{\mathbb{Z}}
\newcommand{\N}{\mathbb{N}}
\newcommand{\G}{\mathbb{G}}
\newcommand{\fr}{\mathfrak}
%\newcommand{\sf}{\mathsf}

% Topology/geometry

\DeclareMathOperator{\Gr}{Gr}
\DeclareMathOperator{\Fl}{Fl}
\DeclareMathOperator{\PP}{\mathbb{P}}
\DeclareMathOperator{\Der}{Der}
\DeclareMathOperator{\Lie}{Lie}
\DeclareMathOperator{\SL}{SL}
\DeclareMathOperator{\GL}{GL}
\DeclareMathOperator{\SO}{SO}
\DeclareMathOperator{\UU}{U}
\DeclareMathOperator{\OO}{O}
\DeclareMathOperator{\Sp}{Sp}
\DeclareMathOperator{\HH}{H}
\DeclareMathOperator{\Symp}{Symp}
\DeclareMathOperator{\Spin}{Spin}
\DeclareMathOperator{\Pin}{Pin}
\DeclareMathOperator{\Td}{Td}
\DeclareMathOperator{\ind}{ind}
\DeclareMathOperator{\vect}{Vect}
\DeclareMathOperator{\Op}{Op}
\DeclareMathOperator{\Maps}{Maps}

% Representation theory

\DeclareMathOperator{\Ad}{Ad}
\DeclareMathOperator{\tr}{tr}
\DeclareMathOperator{\Str}{str}

% Algebra

\DeclareMathOperator{\End}{End}
\DeclareMathOperator{\Aut}{Aut}
\DeclareMathOperator{\Hom}{Hom}
\DeclareMathOperator{\sHom}{\mathscr{H}\!om}
\DeclareMathOperator{\sEnd}{\mathscr{E}\!nd}
\DeclareMathOperator{\id}{id}
\DeclareMathOperator{\irr}{irr}
\DeclareMathOperator{\Diff}{Diff}
\DeclareMathOperator{\gr}{gr}
\DeclareMathOperator{\im}{im}
\DeclareMathOperator{\ad}{ad}
\DeclareMathOperator{\rk}{rk}
\DeclareMathOperator{\Spec}{Spec}
\DeclareMathOperator{\Specm}{Specm}
\DeclareMathOperator{\Stab}{Stab}
\DeclareMathOperator{\Sym}{Sym}
\DeclareMathOperator{\Ext}{Ext}
\DeclareMathOperator{\ch}{ch}
\DeclareMathOperator{\cone}{cone}
\DeclareMathOperator{\cl}{Cl}

% Category theory

\DeclareMathOperator*{\colim}{colim}
\DeclareMathOperator*{\Map}{Map}

\makeatletter
\renewcommand\d[1]{\mspace{6mu}\mathrm{d}#1\@ifnextchar\d{\mspace{-3mu}}{}}
\makeatother

\newcommand{\naturalto}{%
    \mathrel{\vbox{\offinterlineskip
        \mathsurround=0pt
        \ialign{\hfil##\hfil\cr
            \normalfont\scalebox{1.2}{.}\cr
                            %      \noalign{\kern-.05ex}
        $\longrightarrow$\cr}
    }}%
}





\newcommand{\fsl}[1]{{\centernot{#1}}}
\renewcommand\d{\mathsf{D}}

\DeclareMathOperator{\Fun}{Fun}
\DeclareMathOperator{\Emb}{Emb}
\DeclareMathOperator{\Conf}{Conf}
\DeclareMathOperator{\hofiber}{hofiber}
\DeclareMathOperator{\ev}{ev}

\title{Factorization Homology}
\author{John Francis}
\date{Fall 2017}

\begin{document}
\maketitle
\tableofcontents

These are notes from John Francis' ``Factorization homology'' course 
taught during the Fall quarter of the 2017 year at Northwestern. 
Errors and inaccuracies are, as usual, due to the notetaker(s).


\section{What is factorization homology? [09/20/17]}

\subsection{Introduction}

What is factorization homology? Well, if it were an animal, I could describe it in two ways:
distribution and phylogeny. More specifially, we will first see how factorization
homology is distributed over the face of the planet.
Then we will describe how it evolved from single-celled organisms, i.e. how you
might come up with it yourself.

For the moment you can think of factorization homology as a sort of 
\begin{quote}
    \textbf{generalized (co)sheaf homology.}
\end{quote}
Notice that this phrase can be hyphenated in two different ways. In one sense it is a
generalization of the ideas of sheaf cohomology, and in the other it is a homology theory
for generalized sheaves (or sheaf-like objects).
In particular, factorization homology is a machine that takes two inputs:
a geometry $M$ and an algebraic object $A$. The output is
\begin{equation*}
    \int_M A,
\end{equation*}
the factorization homology of with coefficients in $A$.


\subsection{Examples}
Let's look at the first description: what are some examples of factorization homology
that appear naturally in mathematics?
\begin{enumerate}
    \item \textbf{Homology.} Here $M$ is a topological space and $A$ is an abelian group.
        In this case the output is a chain complex
        \begin{equation*}
            \int_M A \simeq \HH_\bullet(M,A),
        \end{equation*}
        quasiisomorphic to singular homology with coefficients in $A$.
        \marginnote{John: What is a theorem you can't prove without ordinary homology?}

    \item  \textbf{Hochschild homology.} Here $M$ is a one-dimensional manifold -- let's
        take in particular $M=S^1$ -- and $A$ will be an associative algebra. In this case
        \begin{equation*}
            \int_{S^1} A \simeq \text{HH}_\bullet A,
        \end{equation*}
        the Hochschild homology of $A$.
        You might be less familiar with this algebraic object than ordinary
        homology. It's importance comes from how it underlies trace methods
        in algebra (e.g. characteristic 0 representations of finite groups).
        Hochschild homology is a recipient of ``the universal trace'' and hence
        an important part of associative algebra. Note that
        $\text{HH}_0A=A/[A,A]$.
        
    \item \textbf{Conformal field theory.} This is in some sense the real starting point
        for the ideas we will develop in this class. Here $M$ is a smooth complete etc.
        algebraic curve over $\C$ and $A$ is a vertex algebra. In this case
        the output $\int_M A$ was constructed by Beilinson and Drinfeld, and
        is known as chiral homology of $M$ with coefficients in $A$. It is
        a chain complex, with $\HH_0(\int_M A)$ being the space of conformal blocks of
        the conformal field theory.

    \item \textbf{Algebraic curves over $\F_q$.} Here $M$ is an algebraic curve over $\F_q$
        and $G$ is a connected algebraic group over $\F_q$. In this case $\int_MG$ is known
        as the Beilinson-Drinfeld Grassmannian and is a stack. One interesting
        property that it has is that
        \begin{equation*}
            \HH_\bullet\left(\int_M G,\bar\Q_\ell\right) \simeq \HH_\bullet(\text{Bun}_G(M),\bar\Q_\ell),
        \end{equation*}
        where here we are taking $\ell$-adic cohomology.
        Although the Beilinson-Drinfeld Grassmannian is more complicated than the
        stack of principal $G$-bundles, it is more easily manipulated.
        We note that the equivalence above is a form of nonabelian Poincar\'e duality.

        In particular, one might be interested in computing
        \begin{equation*}
            \chi(\text{Bun}_G(M)) = \sum_{[P]}\frac{1}{|\Aut(P)|},
        \end{equation*}
        which makes sense over a finite field. The computation of this quantity is
        known as Weil's conjecture on Tamagawa numbers.

    \item \textbf{Topology of mapping spaces.} Now $M$ is an $n$-manifold without boundary and $A$ will be an
        $n$-fold loop space, $A=\Omega^nZ=\Maps( (D^n,\partial D^n),(Z,*))$. The
        output is a space weakly homotopy equivalent to $\Maps_c(M,Z)$ if $\pi_iZ=0$ for $i<n$.
        This is also known as nonabelian Poincar\'e duality. Again the left hand
        side is more complicated but more easily manipulated.

    \item \textbf{$n$-disk algebra (perturbative TQFT).}
        Here $M$ is an $n$-manifold and $A$ is an $n$-disk algebra (or an $E_n$-algebra)
        in chain complexes. The output is a chain complex and has some sort of interpretation
        in physics. One thinks of $A$ as the algebra of observables on $\R^n$, and
        $\int_M A$ is the global observables (in some derived sense). In a rough
        cartoon of physics, one assigns to opens sets of observables, and a way to copmute
        expectation values. Factorization homology puts together local observables to global
        observables:
        \begin{equation*}
            \text{Obs}(M) \simeq \int_M A,
        \end{equation*}
        at least if we are working in perturbative QFT.

    \item \textbf{TQFT.} Here $M$ is an $n$-manifold (maybe with a framing) and $A$ is an
        $(\infty,n)$-category (enriched in $\mathcal{V}$).
        The output is a space (if enriched, an object of $\mathcal{V}$), which
        is designed to remove the assumptions from the examples above.
\end{enumerate}

That's all the examples for now. Next class we'll go over how one might have come up
with factorization homology. It is worth noting that in this class we will focus
on learning factorization homology as a \textbf{tool} instead of aiming to reach
some fancy theorem. Hopefully this will teach you how to apply it in contexts you
might be interested in.

\textbf{Pax:} What is the physical interpretation of the first and second chiral
homologies? \textbf{John:} One might be interested in things like Wilson lines,
where these higher homology groups come into play.

\section{How to come up with factorization homology yourself [09/22/17]}

\subsection{Kan extensions}

Consider the following
thought experiment. Suppose you want to study objects in some context $\mathcal{M}$.
Unfortunately objects here are pretty hard in general. Inside $\mathcal{M}$, however,
we have some objects $\mathcal{D}\subset \mathcal{M}$ that are particulary simple,
and moreover we know that everything else in $\mathcal{M}$ is ``built out of'' objects in $\mathcal{D}$.

Let's consider the example where $\mathcal{M}$ is a nice category of (homotopy types of)
topological spaces. Let $\mathcal{D}$ consist of the point, i.e. all contractible spaces.
Now to study $\mathcal{M}$ we might map functors out of it into some category $\mathcal{V}$.
Let's start with $\mathcal{D}$ instead. Consider $\Fun(*, \mathcal{V})$.
Of course this is canonically just $\mathcal{V}$. How do we extend this to studying $\mathcal{M}$?
We have an obvious restriction map
\begin{equation*}
    \begin{tikzcd}
        \Fun(\mathcal{M},\mathcal{V}) \rar{\ev_*} & \Fun(*, \mathcal{V}).
    \end{tikzcd}
\end{equation*}
We want to look for a left adjoint to this functor $\ev_*$
\marginnote{John: If you don't know what a left adjoint is you should learn it because I won't tell you.
No, I'm not joking (laughs).}
There are two different
things we could do. We could ignore the homotopy-ness of everything, and take the naive
categorical left-adjoint. If, say $\mathcal{V}$ is the category of chain complexes, this naive
left-adjoint produces a stupid answer\ldots depending on what our precise definitions are.
Let's suppose that by $\mathcal{M}$ we meant the homotopy category of spaces (here objects are spaces
and maps are sets of homotopy classes of maps).
Then we are extending
\begin{equation*}
    \begin{tikzcd}
        *\rar{A}\dar & \mathcal{V}\\
        \mathsf{hoSpaces}\urar[dashed]
    \end{tikzcd}
\end{equation*}
A naive left adjoint would take a space $X$ to the stupid answer $A^{\oplus \pi_0X}$.
\marginnote{Why?}
Similarly if we take $\mathcal{M}$ to be just spaces and all continuous maps, $X$ would
be sent to $A^{\oplus X}$. Here by $X$ we mean the underlying set of elements of $X$.

There is a more sophisticated notion of a derived or homotopy left adjoint.
Suppose now that by $\mathcal{M}$
we mean the topological category of spaces, where the mapping sets are spaces equipped
with the compact-open topology. Now we take a \textit{homotopy} Kan extension. This
fancy left adjoint will now send a space $X$ to the the chain complex $C_*(X, A)$
(up to equivalence). Hence we see that we can recover homology from this paradigm
of extending a simpler invariant to the whole category.

How do we choose what $\mathcal{D}$ and $\mathcal{M}$ are? Suppose we want to study
$\mathcal{F}(M)$ for $M\in\mathcal{M}$. For concreteness, let's say we're studying
manifolds. The most basic question to ask: is there a local-to-global principle for
$\mathcal{F}$? The simplest case is for $\mathcal{F}$ to be a sheaf, i.e.
\begin{equation*}
    \begin{tikzcd}
        \mathcal{F}(M) \rar{\sim} & \lim_{U\in\mathcal{U}} \mathcal{F}(U).
    \end{tikzcd}
\end{equation*}
If so you don't need factorization homology, and you can just leave.

For instance, consider $\mathcal{F}=C_*(\Maps(\cdot, Z))$ taking spaces to chain
complexes. Is this a sheaf? Well if we forget about $C_*$, we get a sheaf, as a
map into $Z$ is the same as giving maps on subsets of the domain that agree on overlaps.
What does
taking chains do? Well notice that
\begin{align*}
    C_*(\Maps(U\sqcup V,Z)) &= C_*(\Maps(U,Z)\times\Maps(V,Z))\\
    &= C_*(\Map(U,Z)) \otimes C_*(\Maps(V,Z)).
\end{align*}
This is not a sheaf because in this case tensor products and direct sums are never
the same for these chain complexes! \marginnote{Why?}
So what can we do? We need to change what we consider $\mathcal{D}$ to be from
open coverings to something else.

\textbf{Idea:} to study $\mathcal{F}$ maybe there are more general arrangements of
$\mathcal{D}\subset\mathcal{M}$ such that local-to-global principles still apply, without
$\mathcal{F}$ being a sheaf.

\subsection{Manifolds}

The following problem will guide us for the next few weeks.
\begin{quote}
    Let $M$ be a manifold and let $Z$ be a space.
    Calculate the homology of the mapping space $\HH_\bullet\Maps(M,Z)$.
\end{quote}

To begin, let us specify which categories we will be working with.
\begin{definition}
    Let $\mathsf{Mfld}_n$ be the (ordinary) category of smooth $n$-manifolds, with
    $\Hom(M,N)=\Emb(M,N)$ the set of smooth embeddings of $M$ into $N$.
    Similarly, 
    let $\mathcal{M}\mathsf{fld}_n$ be the \textit{topological} category of smooth $n$-manifolds, with
    $\Hom(M,N)=\Emb(M,N)$ the \textit{space} of smooth embeddings of $M$ into $N$, equipped with the
    compact open smooth topology.
\end{definition}

The compact open smooth topology takes a bit of work to define, so we'll leave that
as background reading. A good reference is Hirsch's book on differential topology \cite{hirsch}.
Roughly, convergence in this topology is pointwise in the map as well as all its derivatives.
To get a feel for what this entails, consider a knot. Locally tighten the knot
until the knot turns (locally) into a line. These knots would would converge in the usual compact-open
topology to another knot, but in the smooth topology, they do not converge as the
tightening procedure creates sharp kinks.
In particular $\pi_0\text{Emb}(S^1,\R^3)$ is very different from
$\pi_0\text{Emb}^\text{top}(S^1,\R^3)$.

\begin{definition}
    We define the category $\mathsf{Disk}_n$ to be the full subcategory of
    $\mathsf{Mfld}_n$ where the objects are finite disjoint unions of standard
    Euclidean spaces $\coprod_I\R^n$. Similarly the category $\mathcal{D}\mathsf{isk}_n$
    is the full \textit{topological} subcategory of $\mathcal{M}\mathbf{fld}_n$ where the objects
    are finite disjoint unions of Euclidean space.
\end{definition}

Observe that $\Hom_{\mathcal{D}\mathsf{isk}_n}(\R^n,\R^n)=\text{Emb}(\R^n,\R^n)$.
\begin{lemma}
    The map $\Emb(\R^n,\R^n)\to GL_n(\R)\simeq O_n\R$ given by differentiating
    at the origin is a homotopy equivalence.
\end{lemma}
\begin{proof}[Proof sketch]
    There is an obvious map $GL_n\R \to \Emb(\R^n,\R^n)$. One of the composites is
    thus clearly the identity. It remains to show that the other composition is
    homotopic to the identity. The homotopy is given by shrinking the embedding
    down to zero.
\end{proof}

This fact should fill you with hope. The objects which are building blocks of manifolds
have automorphism spaces that are, up to homotopy, just finite-dimensional manifolds.
Actually it will be useful to think of the $n$-disks as some sort of algebra.

\begin{definition}
    An $n$-disk algebra in $\mathcal{V}$ is a symmetric monoidal functor
    $A:\mathcal{D}\mathsf{isk}_n\to\mathcal{V}$.
\end{definition}

As we stated before, our first goal in this class is to understand the homology 
$\HH_*\Maps(M,Z)$ using $n$-disk algebras and factorization homology.

Question from someone: what's the relation with $E_n$-algebras? John: It turns
out that $E_n$-algebras are equivalent to $n$-disk algebras with framing.

Question from Tochi: what if you work with manifolds with boundary? John: well
if you require boundaries to map to boundaries you can make the same definitions.
You then have to work with Euclidean spaces and half-spaces. You'll end up
with two types of algebras instead of just $n$-disk algebras.

\section{Framed embeddings [09/25/17]}

\begin{definition}
    A \textbf{framing} of an $n$-manifold $M$ is an isomorphism of vector bundles
    $TM \cong M\times \R^n$.
\end{definition}

Of course, not all manifolds have framings. For instance, one
can check that all (compact oriented) two-manifolds except for $S^1\times S^1$ do not admit framings.
You might use the Poincar\'e-Hopf theorem, which expresses the Euler characteristic
as a sum of the index of the zeroes of a vector field $v$ on $M$ that has isolated
zeroes. Hence if $M$ is framed, the Euler characteristic of $M$ must be zero.

Here is an example of a theorem that John does not know how to prove without the use of homology.
\begin{theorem}[Whitney or Wu]
    Every orientable three-manifold admits a framing.
\end{theorem}
Pax: isn't there a later proof of this via geometric methods by Kirby? John: well ok
I didn't mean it as a formal statement that it can't be proved without\ldots

Notice that any Lie group has a framing, as one takes a basis for the Lie algebra
and pushes it forward by the group action. On the other hand, manifolds of dimension
four generally do not have framings (at least in John's experience).

We can ask the following question: what is a framed open embedding? There are a few options.
The naive (strict) option is that if we have an open embedding $M\hookrightarrow N$ of
framed manifolds, since the pullback of $TN$ is $TM$, we have two different trivializations
of $TM$. We might ask that the map induced $M\times\R^n\to M\times\R^n$ is the identity.

Okay, let's think about framed embeddings. Embeddings are very flexible -- you can stretch
them and twist them. But strict framed embeddings are very rigid. For instance, they are
automatically isometries (giving the fibers the usual Euclidean metric). But there aren't
very many isometric embeddings into a compact manifold. So this options is not what we
will be interested in working with.

Let's consider a more lax/homotopy-theoretic option. Thinking slightly differently, recall
that the tangent bundle is classified by a map $TM: M\to \Gr_n\R^\infty$. This map is
of course only defined up to homotopy. That's fine, just choose a representative. Over
the infinite Grassmannian we have the Stiefel manifold $V_k(\R^\infty)\to\Gr_n\R^\infty$.
Choosing a lift
\begin{equation*}
    \begin{tikzcd}
        \; & V_n\R^\infty \dar \\
        M \rar{TM}\urar[dashed]{\phi_M} & \Gr_n\R^\infty
    \end{tikzcd}
\end{equation*}
is precisely the data of a framing. Suppose now that an embedding $M\hookrightarrow N$
where $M,N$ are framed by $\phi_M$ and $\phi_N$.
The lax definition of a framed embedding is now going to be extra data: an embedding
together with a homotopy between the framings $\phi_M$ and $\phi_N|_M$.


\begin{definition}
    The space of framed embeddings $\Emb^{fr}(M,N)$ is the homotopy pullback
    \begin{equation*}
        \begin{tikzcd}
            \Emb^{fr}(M,N) \rar\dar & \Emb(M,N)\dar  \\
            \Maps_{V_n\R^\infty}(M,N) \rar & \Maps_{\Gr_n\R^\infty}(M,N)
        \end{tikzcd}
    \end{equation*}
    In particular a framed embedding is an embedding $M\hookrightarrow N$ and
    a homotopy in $\Map_{\Gr_n\R^\infty}(M,N)$ between the images along both maps.
\end{definition}

\begin{exercise}
    Check that $V_n\R^\infty\simeq *$.
\end{exercise}

It looks like we've made things more complicated, whereas framings should have made
things simpler. Let's check that it is. Let's calculate $\Emb^{fr}(\R^n, \R^n)$.
By definition, this sits in the following diagram
\begin{equation*}
    \begin{tikzcd}
        \Emb^{fr}(\R^n, \R^n) \rar\dar & \Emb(\R^n, \R^n)\dar \\
        \Maps_{V_n\R^\infty}(\R^n, \R^n) \rar & \Maps_{\Gr_n\R^\infty}(\R^n,\R^n).
    \end{tikzcd}
\end{equation*}
Notice that the bottom left object is homotopy equivalent to $\Maps_*(\R^n,\R^n)\simeq *$.
The bottom right space is homotopy equivalent to the loop space $\Omega\Gr_n\R^\infty\simeq \Omega BO(n)\simeq O(n)$.
From last time, $\Emb(\R^n,\R^n)\simeq \Diff(\R^n)\simeq GL(n)\simeq O(n)$ (this is
\textbf{homework 1}). Now the vertical map on the right is a homotopy equivalence.
This implies (by some machinery) that the vertical map on the left is an equivalence.
We conclude that $\Emb^{fr}(\R^n,\R^n)\simeq *$. The rest of \textbf{homework 1} is to
show that $\Emb(\R^n, N)$ is homotopy equivalent to the frame bundle of $TN$. Applying this
to the diagram above where we replace the second copy of $\R^n$ with $N$, we obtain
\begin{equation*}
    \begin{tikzcd}
        \Emb^{fr}(\R^n, N) \rar\dar & \Emb(\R^n, N)\dar \\
        \Maps_{V_n\R^\infty}(\R^n, N) \rar & \Maps_{\Gr_n\R^\infty}(\R^n,N).
    \end{tikzcd}
\end{equation*}
Now the same argument will show that the vertical map on the right is an equivalence,
and that the map of the left is an equivalence. It follows now that $\Emb^{fr}(\R^n,N)\simeq N$.
Hence we see that we are replacing the role of the orthogonal group by that of a point.

The point of adding these framings is that it will allow for an easier transition between
algebra and topology.
\begin{definition}
    We define the category $\mathbf{DISK}_n^{rect}$ to be a topological category with objects
    finite disjoint unions of open unit disks, with morphisms being embeddings that are compositions
    of translations and dilations. These embeddings are what we call rectilinear embeddings.
\end{definition}
These embeddings are easy to analyze (we've given the space the obvious space of pointwise
convergence). This space of embeddings from a single disk to a single disk is a contractible
subspace of $D^n\times\R_{>0}$ (just move to the origin and expand outwards). More generally,
one checks that rectilinear embeddings $\Emb^{rect}(\sqcup_k D^n, D^n)$ maps to
$\Conf_k(D^n)$ and that this map is a homotopy equivalence.

\begin{proposition}
    There is a homotopy equivalence $\mathbf{DISK}_n^{rect}\simeq \mathbf{DISK}_n^{fr}$.
\end{proposition}


\section{Homotopy pullbacks and framing [09/27/17]}

\begin{definition}
    Suppose we have a map $X\to B$ together with a point $*\in B$, the homotopy fiber
    of $X\to B$ over $*\in B$ is the fiber product $\{*\}\times_B\Maps([0,1], B)\times_B X$.
\end{definition}

\begin{lemma}
    The formation of homotopy fibers is homotopy invariant. More precisely, given an weak
    equivalence of spaces $X\to X'$ over $B$ a pointed space via maps $f$ and $g$,
    then the homotopy fiber of $f$ is weak equivalent to the homotopy fiber of $g$.
\end{lemma}
\begin{proof}
    We can use the long exact sequence of a Serre fibration.
    \begin{equation*}
        \begin{tikzcd}
            \hofiber(f)\dar \ar[rr] & \; & \hofiber(g)\dar \\
            \Maps([0,1],B)\times_B X \ar[rr]\drar & \; & \{*\}\times_B\Maps([0,1],B)\ar[dl] \\
            \; & B & 
        \end{tikzcd}
    \end{equation*}
    Some argument about homotopy equivalences allows us to conclude that $\pi_*\hofiber(f)\cong\pi_*\hofiber(g)$.
\end{proof}


\textbf{Homework 2}: homotopy pullbacks are homotopy invariant

Recall last time we were discussing $\Maps_B(M,N)$ for some space $B$. This objects
is defined to be the homotopy fiber
\begin{equation*}
    \begin{tikzcd}
        \Maps_B(M,N) \rar\dar & \Maps(M,N) \dar \\
        * \rar & \Maps(M,B)
    \end{tikzcd}
\end{equation*}
Now by homotopy invariance we can argue that $\Maps_{V_n\R^\infty}(M,N)\simeq \Maps(M,N)$
since $V_n\R^\infty\simeq*$. Hopefully this fills in some of the gaps from last lecture.

Let us now return to our assertion from last time.
\begin{proposition}
    There is a functor $\mathbf{DISK}_n^{rect}\to \mathbf{DISK}_n^{fr}$ which is a homotopy
    equivalence.
\end{proposition}
\begin{proof}
    Notice that we can just calculate from last time,
    \begin{equation*}
        \Emb^{fr}(\R^n, \R^n) \simeq * \simeq \Emb^{rect}(D^n, D^n).
    \end{equation*}
    More generally, consider $\Emb^{rect}(\sqcup_I D^n, \sqcup_J D^n)=\sqcup_{\pi:I\to J}\prod_J\Emb^{rect}(\sqcup_{\pi^{-1}(j)}\R^n,\R^n)$.
    So it suffices to show that $\Emb^{fr}(\sqcup_I\R^n,\R^n)\simeq\Emb^{rect}(\sqcup D^n, D^n)$.
    Recall that $\ev_0:\Emb^{rect}(\sqcup_I D^n,D^n)\to \Conf_I(D^n)$ is a homotopy
    equivalence, which we argued last time. Returning to our homotopy pullback square
    \begin{equation*}
        \begin{tikzcd}
            \Emb^{rect}(\sqcup_I \R^n, \R^n) \rar\dar & \Emb(\sqcup_I\R^n,\R^n)\dar \\
            *\simeq \Maps_{EO(n)}(\sqcup_I \R^n,\R^n) \rar & \Maps_{BO(n)}(\sqcup_I \R^n,\R^n)
        \end{tikzcd}
    \end{equation*}
    notice that
    \begin{equation*}
        \begin{tikzcd}
            \text{Fr}(TM)\simeq \Emb(\R^n, M)\dar{\ev_0} & \Emb( (\R^n,0), (M,x))\lar\dar \simeq O(n) \\
            M & \{x\}\lar
        \end{tikzcd}
    \end{equation*}
    Likewise
    \begin{equation*}
        \begin{tikzcd}
            \Emb(\sqcup_I \R^n, M)\dar{\ev_0} & \prod_I O(n)\lar \dar\\
            \Conf_I(M) & \{x_1,\ldots,x_I\}\lar
        \end{tikzcd}
    \end{equation*}
    Hence $\Maps_{BO(n)}(\sqcup\R^n,\R^n)\simeq \prod_I\Maps_{BO(n)}(\R^n,\R^n)\simeq\prod_IO(n)$.

    Up to homotopy, we now obtain
    \begin{equation*}
        \begin{tikzcd}
            \Emb^{fr}(\sqcup\R^n,\R^n) \rar\dar & \Conf_I(\R^n)\times \prod_I O(n)\dar \\
            * \rar & \prod_I O(n)
        \end{tikzcd}
    \end{equation*}
    so we conclude that $\Emb^{fr}(\sqcup_I\R^n,\R^n)\simeq \Conf_I(\R^n)$ which concludes
    the proof of the proposition.
\end{proof}

\begin{example}
    Consider the case $n=1$. What do the framed and rectilinear embeddings look like in this case?
    Notice that $\Emb^{fr}(\sqcup_I \R^1,\R^1)\simeq \Conf_I(\R^1)$ is discrete up to
    homotopy, and identified noncanonically with the symmetric group on $I$ letters.
\end{example}

Recall a definition from the first day.
\begin{definition}
    An $\mathcal{E}_n$ algebra in $\mathcal{V}$ is a symmetric monoidal functor
    $\mathbf{DISK}_n^{rect,\sqcup} \to \mathcal{V}^\otimes$.
\end{definition}
Next time we will see that $\mathcal{E}_1$-algebras are, in a suitable sense,
equivalent to associative algebras.





\bibliographystyle{alpha}

\bibliography{references}



\end{document}

